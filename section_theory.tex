%\clearpage

\section{Theory of ice slurry and supercooling}
\label{chapter_theory}

Ice slurry is a mixture of small ice crystals or particles and a carrier fluid. Ice crystals are in the size of 0.1 to 1 mm; when they are larger than 1 mm they are usually referred to as ice particles \citep{kauffeld_ice_2010}.
Usually, the carrier fluid is either water or a mixture of water and a freezing point depressant, e.g. sodium chloride, ethanol, ethylene glycol and propylene glycol. 
%In the present work demineralized water is focused on in this work since it is cheap, highly available and not harmful to the environment. 
%Tap water as a carrier fluid will be focused in a later stage due to simplicity and availability.


%\subsection{Classical nucleation theory} %(\acs{CNT})
The classical nucleation theory \citep{turnbull_rate_1949} explains that the formation of ice, i.e. the phase change from liquid to solid form, occurs via a metastable state that can be reached via \emph{supersaturation}. This supersaturation can be achieved either with pressure, concentration and/or temperature variations (\emph{supercooling}) \citep{kauffeld_handbooks_2005}. In this metastable state, small clusters of water molecules, \emph{nuclei}, are formed due to statistical thermal spatial fluctuations. The following nucleus growth is a result of a series of equilibrium mechanisms that correspond to the local free energy of nucleation, which in return is related to the free energy of activation for the short-range diffusion at the phase transition interface \citep{turnbull_rate_1949}. If thermal conditions favor growth of the nucleus, its size may increase beyond the critical radius. Further growth above the critical radious releases energy and spontaneous growth occurs. If the fluctuations do not favor growth, the smaller nuclei below the critical radius may decay and eventually dissolve. The creation of nuclei from a purely metastable state is called \emph{homogeneous nucleation} and its formation rate depends exponentially on the degree of supercooling. 

In practice, most nucleation processes are formed due heterogeneous nucleation, which describes nucleation at foreign surfaces. As the free energy of nucleation is the sum of the counteracting parts of the volume free energy and the surface free energy, these counteracting parts behave differently in the presence of foreign surfaces. The resulting heterogeneous work of nucleation  is lower than the homogeneous work of nucleation by a factor that depends on the interface and the geometry of the surface. Heterogeneous nucleation requires lower degrees of supersaturation compared to heterogeneous nucleation. If heterogeneous nucleation occurs at the surface of an already existing seed crystal, it is called \emph{secondary nucleation}.

\subsection{Ice slurry production methods}

The mentioned three steps during creation of an ice slurry, supersaturation, nucleation and growth, can be achieved and controlled in multiple ways. Reported methods include \citep{kauffeld_handbooks_2005, zhang_overview_2012, mouneer_heat_2010}:
\begin{enumerate}
  \item \textbf{Scraper design} includes a moving scraper that periodically or continuously removes the ice crystals from a heat exchanger wall. It it one of the simplest and most common methods, but involves moving parts and requires motor power \citep{ernst_influence_2016}. 
  \item \textbf{Orbital rod generators}
    work similar to the scraper design. A moving whip rod is used to de-ice the surface, which is commonly the inner wall of a tubular heat exchanger. Unlike the scraper design, it does not touch the exchanger walls and usually moves much faster compared to the scrapers.
  \item \textbf{Fluidized bed generators}
    use movings beads whirled inside the fluid flow. These beads randomly hit the heat exchanger's wall, removing the ice crystals that grow on the wall surface.
  \item \textbf{Supercooling method} separates the location of cooling and nucleation. Thus, ice does not grow on the heat exchanger wall surfaces and no moving parts are needed. However, a stable prevention of nucleation at the heat exchanger wall is a challenge.
  \item \textbf{Direct contact heat exchangers} benefits from a large heat transfer between the icing fluid and the immiscible refrigerant which are in direct contact. As the refrigerant transfers heat to the icing fluid, it evaporates and can be then removed from the freezing fluid.
  \item \textbf{Vacuum ice slurry generation} works using states close to the triple point of water. Cooled water is sprayed into a deep vacuum, such that part of the water evaporates. The latent heat of vaporization is removed from the remaining liquid water, which then freezes. 
\end{enumerate}

In this study, the method of supercooling will be used, since it does not require additional energy for e.g. motors, it does not utilize moving parts and is technology readiness level is believed to suit best the foreseen application as a heat source for heat pumps.


\subsection{Ice slurry production by supercooling}
\label{section_lit_review_supercooling_theory}

To be able to create a high degree of supercooling, the formation of ice needs to be prevented as long as possible. While the desired high degree of supercooling is the main driver for crystallization, the likeliness of formation of ice can be lowered by influencing the free energy of nucleation. 
%The most common and simplest method of lowering the crystallization temperature is the use of additives. For this project, it has been decided not to make use of additives, since these are a further factor of cost, alter various other properties of water and can cause environmental issues. 

The supercooling method suffers from unstable operation, and this has limited further market deployment. 
One of the challenges is to avoid or reduce the formation of ice on the surface of the supercooler heat exchanger. 
A second challenge is to provoke ice formation when desired by promoting ice nucleation \citep{beaupere_nucleation_2018} and controlling the ice concentration for proper distribution in the ice storage.
Undesirable ice growth in the coldest part of the heat exchangers may cause blockages, and although the process is self-sealing, the refrigerant cycle operation is stopped and an extra energy and supply system for melting the ice formed is necessary. 

Ice nucleation is stochastic, i.e. ice will not nucleate at the same temperature and time during a cooling process on apparently identical surfaces; however, the probability of nucleation increases {\em exponentially} with the degree of water supercooling. Also, the probability of freezing increases with the time that supercooled water is in contact with a surface. 
Therefore, a safe margin of the degree of supercooling and duration of ice slurry production needs to be considered in order to achieve a high probability of nucleation suppression.
The availability of nucleation sites and the probability of nucleation can be reduced using the following methods. 

\subsubsection{Roughness of the nucleation surface}

Methods to reduce the probability that ice nucleates on, grows on, and adheres to a surface -- and thus to increase the supercooling degree and long-term performance of the system -- include reducing the roughness of the surface, which reduces the area and the likelihood of nucleation. Applying a hydrophobic coating to the surface and thus reducing the ice adhesion is a common approach to increase the degree of supercooling.
\cite{faucheux_influence_2006} analyzed the influence of aluminum roughness on the supercooling degree under static conditions, concluding that it increases when surface roughness decreases.
%of an aqueous solution with different concentrations of ethanol.
%follow the equation $\Delta T_{sc}=7.15 r_a^{-0.196}$, where $r_a$ is the roughness in $\mu m$.
Roughness between \SIrange{0.63}{13.33}{$\mu$m} were analyzed leading to 8.5~K and 4.2~K supercooling degrees correspondingly. Thus, lowering the surface roughness was found to increase the supercooling degree significantly. The suspected relationship between supercooling degree and antifreeze concentration was not experimentally confirmed.
These investigations showed very high supercooling degrees, but are not entirely applicable to continuous processes since they were obtained using static conditions.
\cite{ernst_influence_2016} analyzed the influence of aluminum roughness on the supercooling degree under steady flow conditions of \SI{265}{kg/h} and roughness of \SIrange{0.4}{2.1}{$\mu$m}. They reported a maximum supercooling degree with a continuous cooling process of \SI{3}{K}.  With higher supercooling degrees, the pipe was blocked by ice formation. They confirmed the findings of \cite{faucheux_influence_2006} and explained the lower supercooling degrees due to fluid flow conditions and discussed the possible effect of impurities in the used water.


\subsubsection{Material and coating of the surface}

A common approach to avoid or delay ice nucleation is to use hydrophobically coated surfaces. Lower wetted contact area of water and the adjacent surface increases the surface free energy, reducing the probability of nucleation at the surface. 
\cite{saito_fundamental_1994} investigated the effect of the heat transfer surface characteristics on the freezing of supercooled pure water. It was observed that the supercooling degree was highly dependent on the characteristics of the surface. The oxidation of a polished copper surface was able to prevent ice growth on the surface.

\cite{jung_are_2011} report that surfaces with very low roughnesses in the range of the critical nucleation radius and high wettability display longer freezing times compared to superhydrophobic surfaces with a very low wettability, although these surfaces are also reported to reduce ice formation. The property of a high water contact angle of superhydrophobic surfaces seems to be competing with the desired low roughness. They conclude that superhydrophobicity does not necessarily need to to be equal to \emph{icephobicity} and suggest finding an optimum between wettablilty and low roughness. 
These findings were confirmed by \cite{hejazi_superhydrophobicity_2013}, who suggests to widen the expression of icephobicity to include the property of low adhesion strength of ice to the surface to facilitate the removal of ice. \cite{janjua_performance_2017} suggest that the adhesion strength of ice to a surface is lowered with high receding \emph{contact angle} and low contact angle hysteresis.

\cite{kauffeld_handbooks_2005} describe an influence of the \emph{lattice structure} of the nucleating surface on the surface and volume free energy. A lattice mismatch between the ice and the nucleating surface induces a strain into the nucleus, which increases the bulk free energy of the nucleus. Consequently the nucleation efficiency and growth speed are lowered. \cite{qiu_ice_2017} confirmed these findings by means of simulation and concluded that surface fluctuations decrease its ice-freezing efficiency.  

%flow influence
The effects of fluid flow regime on supercooling degrees is still unclear \citep{kauffeld_ice_2019}. 
However, from classical nucleation theory follows that the mass transfer governed part of the crystallization process and crystal growth is slowed down if the kinetic movement of the liquid is lowered \cite{ronceray_suppression_2017, turnbull_rate_1949, kelton_nucleation_2010}.
In \cite{Luo_homogeneous_2020,Luo_ice_2019}, the effect of shear flow on ice formation is simulated. It is concluded that the shear rate influences different mechanisms. Increase of the shear stress reduces the stability  and the growth of small nuclei. At the same time, the existence of a shear stress increases the diffusion and the formation of nuclei. At low shear rates, the increased generation of nuclei dominates. As the shear rate increases, the exponential relation to the increased energy barrier for nucleation dominates and reduces the nucleation rate. However, these findings are strongly dependent on the supercooling degree, which dominates the aforementioned effects. Similar behaviors were reported for ice growth rates. Again two effects compete: At low shear rates, shear mainly breaks the hydrogen bond network of water, allowing water molecules to reorganize and bond with the existing ice. At higher shear rates, shear predominately disrupts the water-ice hydrogen bonds, reducing the growth of ice. At temperatures close to the freezing point, these effects are dominated by thermal effects.



%bedecarrats 2010: possible to find state with enough safteey margin to build someting useful




\subsection{State of the art on supercooling}
\label{section_lit_review_supercooling}

There is a significant gap between Japan and the rest of the world on the know-how and state-of-the-art of ice slurry technologies, especially supercooling-based methods. Already in 2001, the number of ice slurry systems in operation was far larger in Japan than in any other country of the world \citep{kauffeld_ice_2019}. It is likely the large difference between the night/day electricity tariffs that pushed several players on the Japanese market. 
%In most other countries there has only been one single supplier at a time trying to convince their customers of this new technology based on an inefficient and costly scraper type. 
There are at least three companies in Japan that commercialize ice slurry systems using the supercooling method: Takasago Thermal Engineering, Mayekawa  and Shinryo Corporation. These companies use large heat exchangers to overcompensate the instability of the supercooled fluid and the partly blocked heat exchangers. At a certain point even these systems need to be shut down and thawed, which is obviously inefficient. Some literature regarding state that a degree of supercooling of 2 K is achieved.
However, the real metrics of freezing times and technical difficulties are not communicated. It is worth noticing that these companies do not operate at all in Europe, so it is not possible to use or test any of their systems here.
 
There are several publications from the research group of the Japanese company Takasago Thermal Engineering where they explain the concept of supercooling being applied for district cooling networks.
Researchers from this company presented an ice slurry generator with a capacity of 13~kW with 2~K of supercooling  \citep{tanino_ice-water_2001,kozawa_study_2005}. Unfortunately, no details on the heat exchanger type or surface treatments used in the supercooler were given. Moreover, nothing was mentioned about the stability, reliability and icing frequency of the system. Results that experimentally prove, that a steady-state 2~K degree of supercooling can be achieved, were also not given. 


The state-of-the-art in Europe, and in general everywhere else except Japan, is far below in the Technology Readiness Level (TRL) chain. 
\cite{castaing-lasvignottes_dynamic_2006} and \cite{bedecarrats_ice_2010} investigated numerically and experimentally a supercooler heat exchanger, where stable operation was achieved with 1.2 K of supercooling, leading to slow ice production and only 1.3~kW cooling power.
%The coaxial coil heat exchanger had an internal diameter of 11 mm %and 17.5 mm for the inner and outer tube respectively.
%Wall thicknesses of 0.7 mm were used for both tubes whose length was 5 m. Tap water without any additives was circulated in the annular zone. 
In their experiments, supercooling of 2~K was found to be unstable. The heating power of 1.3~kW for this low supercooling degree was not enough to completely evaporate the refrigerant and a second evaporator was used for this purpose. %Mass flow rates used in the experiments were in the order of 540 - 650 kg/h. Tap water and heat exchanger without any special treatment were used. 
The same heat exchanger was used in following experiments by \cite{bedecarrats_ice_2010}. On this new experiment, a maximum supercooling degree was found to be 2~K, but blocking was very frequent. For supercooling degrees below 1.2~K there was no blocking but ice production was very slow. Between 1.3~K and 1.9~K there were regular blockages (less than 2 per hour).
It was found that the supercooling degree increased by lowering the flow rate. However, higher flow rates produce more ice, thus an optimum exists.
It was found that for the same supercooling degree, higher flow rates lead to higher number of blockages which was explained with the assumption that turbulence promotes cristallisation. 
%However, event that our intuition might agree on this argument, to our understanding there is no scientific proof that the fluid regime affects ice nucleation. 
%The largest supercooling degree corresponding for a stable operation was approximately of \SI{1.8}{K} for a flow rate of \SI{0.12}{kg/s}, but only of 0.9~K for a flow rate of 0.18~kg/s. 
%An optimal operation was found for 0.14~kg/s and a supercooling degree of 1.6~K. 
% WE NEED VELOCITY VALUES NOT MASS FLOW. 
These experiments showed that achieving stable operation with a decent degree of water supercooling -- which we describe summarily as \textbf{\em supercooling degree} -- is challenging, and without any surface treatment, it does not seem to be possible.


During analyses of the  roughness of a coaxial copper tube with supercooled flows, supercooling data was provided by \cite{ernst_influence_2016}. 
Tap water was supercooled in a coaxial tube heat exchanger with an inner diameter of 8 mm, a length of 5~m and a total area of 0.24~m$^2$ with a  cooling capacity of around 1.2~kW.
Using a mass flow rate of 265~kg/h the maximum degree of supercooling was 3~K.
Roughness was varied between 0.4~\si{$\mu$m} and 2.1~\si{$\mu$m} and the degree of supercooling was 2.3~K and 3~K, respectively, using a water velocity of 0.36~m/s.
%The supercooling degree was found to follow a linear relationship:
%\begin{equation}
%\Delta T_{sc}=3.1 - 0.7 r_a \hspace{0.5cm} r_a [0.4-2.1] \mu m
%\end{equation}
The supercooling degree was higher than the ones reported by \cite{bedecarrats_ice_2010} and \cite{castaing-lasvignottes_dynamic_2006}.
While the frequency of ice blockages as a function of the maximum supercooling degree that was achieved was not stated in the paper, the authors reported by personal communication that the supercooling degree was stable over several hours. 
However, these results were obtained using a polished long tube (around 5~m) with gentle bends and only 1~kW of cooling capacity. This kind of heat exchanger is not suitable for any practical application.


In \cite{wang_experimental_2012}, the authors coated  a coaxial tube heat exchanger (12 mm internal diameter) with a polymer that contained a fluoric binder and an organic solvent. They observed an increase in the maximum supercooling degree from 0.9~K to 1.7~K at a velocity of 2~m/s thanks to the fluorocarbon coating. However, even with the coated surface, the highest supercooling degree only lasted 9 minutes. Increasing the velocity to 2.5~m/s lead to frequent blockage by the ice growth.
It was also observed that the maximum supercooling state lasted 6 minutes for the uncoated surface and 9 minutes for the coated one when tap water was used.
%Pure water has a higher supercooling degree compared to tap water because it has less impurities that could initiate nucleation. 
The maximum supercooling degree was obtained with velocities of 2~m/s. Values below 1.5~m/s were not able to produce any slurry and above 2.5~m/s the heat exchanger was frequently blocked by the growth of ice.
%The heat exchanger was a co-axial tube with 12 mm internal diameter, 1 mm thickness of the inner tube, and 22 mm internal diameter and 1.5 mm thickness of the outer tube.
To address the problem of not achieving continuous ice production  due to ice blockage, the same authors in \cite{wang_investigation_2016} proposed a double uncoated supercooler concept, such that once an ice blockage exists in one supercooler, a second one would operate. In this paper, a corrugated flat plate was used with a degree of supercooling in the range of \SIrange{0.5}{1}{K} with low fluid velocities of \SIrange{0.1}{0.45}{m/s}. In practice a flat plate heat exchanger would operate above 1~m/s. 
%\textbf{\em To the author's knowledge this is the only work that reports experimental proof of the real supercooling degree that was achieved while using an industrially relevant heat exchanger. While true, the low 1~K supercooling degree was achieved with low water velocities, both of which reduce the cooling capacity considerably.}


%\subsubsection{Conclusions on the state of the art of supercooling}

%Summary of the review. Achievements on supercooling power. what has been proof


\subsection{Ice crystallizer - supercooling release methods}
\label{section_lit_review_ice_crystallizer}


\subsubsection{State of the art of crystallization methods}
Controlled release of the supercooled state of the water allows to adjust ice particle size, slurry properties and to prevent ice from blocking pipes through icing. Crystallizers should therefore be close to the supercooler and should allow for complete release of the supercooling degree. 




\begin{comment}
Early crystallizing devices used the simple method of seeding or physical disturbances as supercooling release methods \citep{bedecarrats_supercooled_2000}. The used systems however had multiple drawbacks. Seeded methods fail to operate if the temperature of the system rises such that the initial seed crystal completely melts. A new seed crystal then has to be added to the system in order to start the controlled crystallization process again. Also, seed crystals might get flushed away if they are located inside a device in the pipe \citep{kristallisationseinheit_karlsruhe}. The early models using physical disturbances used to be open to the environment \citep{tanino_ice-water_2001}. Open systems suffer from added impurities from the surroundings, which can be a major drawback if foodstuff should be cooled with the slurry. Reproducibility is also lower with this kind of systems. \cite{saito_fundamental_1994} tested various methods like mechanical shock, rubbing of solid surfaced, electrical shocks, vibration, concluding that different kinds of shock may start nucleation with increased tendency for increased collision force. In the early 2000s, more sophisticated approaches used ultrasonic waves to start nucleation. This method allows controlled nucleation in closed pipes, is reported to prevent adhesion of ice in the are of the crystallizer and to allow adjustment of ice crystal size. This method requires some initial energy, but can be turned off as soon as enough present ice crystals can act as seed crystals. Recently, more research is directed to magnetic and electric disturbances to use for controlled nucleation \citep{dalvi-isfahan_review_2017}.
Main requirements for crystallizing devices seem to be the controlled and continuous crystallization, complete release of the supercooling degree of the fluid, prevention of upstream propagation of ice and a closed design to prevent impurities from entering the system. 

\end{comment}



\noindent Except for crystallization through seeding, four approaches are reported \citep{wang_investigation_2016, zhang_overview_2012}:

\begin{enumerate}
  \item Collision of the supercooled fluid with a different surface. This may be a solid, a free liquid surface in a tank or a collision of multiple supercooled flows \citep{bedecarrats_ice_2010}. Crystal agglomeration near the impact place needs to be considered. \cite{wang_investigation_2016} used a 400-mesh sieve filter with \SI{37}{$\mu$m} apertures.  
  \item Ultrasonic vibrations are reported to be an effective means of controlled release of the supercooled state. Acoustic cavitation, i.e. the sudden formation and collapse of gas bubbles in liquids by means of ultrasound, seems to cause the nucleation of ice \citep{baillon_28_2015}. The so-called \emph{sonocrystallization} typically results smaller but larger number of crystals compared to crystallization without an ultrasound. 
  \item Locally increasing the supersaturation from the metastable to the labile region increases the probability of nucleation. However, this threshold is, among others, strongly depending on the supersaturation rate and set-up \citep{mullin_crystallization_2001}. Therefore, this method can be used e.g. by utilisation of small and local electrical cooling or by changing the properties of the pipe. \cite{le_bail_ice_2015} use different types of surface roughness over the length of their tubular heat exchanger to promote crystallization. They report continuous production of ice slurry using a water-ethanol mixture, but also state a big impact of the flow velocity on the crystal growth pattern. They achieved dendritic crystal growth, resulting in ice slurry plugs that were pushed out of the heat exchanger. Different material within heat exchangers with varying heat transfer rates could also help nucleation.
  \item The influence of electric and magnetic fields in nucleation was repeatedly researched. Although it was shown that both types of field influence the supercooling degree, the crystallization rate and the quality of crystallization, the mechanisms do not seem to be fully understood \citep{dalvi-isfahan_review_2017}.
\end{enumerate}


%wang_effective_2014 trief with coated surfaces for supercoolers, but switched to a double-HX-relaxation device wang_investigation_2016

%Beaupere 2018 nucleation: review of different nucleation methods: unbedingt einbauen!


%ice growth speed and theory: Thammann et al 1935 crystallization speed




%\subsection{State of the art on ice slurry storage design}
%\label{section_lit_review_iceslurry}

%oechsle 2016 eisspeicher (DKV tagung): stand der technik zu speicher und beladung






% various useful references

%Zhang 2011: Performance improvement of vertical ice slurry generatorby using bubbling device; prevent ice from sticking to walls in shell and tube heat exchanger

%Liu 2016: we will always stick to the stationary bed flow type